\chapter{Section 2
The making of an Embodied mandala}


%Repetition- Title: I understand it all better

%Mindfulness- Titled: Going within, but allowing it back out

%Processing- Titled: Unlocking the vault

%Hypnotic focusing- Titled: Be ok in it

%Decelerate/calming- Title: No words

%Connection/awareness- Titled: Grounding connections 

The mandala process I work from is my own, and before I move into my process I want to make comment on the phenomena of mandala colouring books. I find this somewhat disconcerting, although I concede that colouring in may support relaxation and from many accounts form of stress relief. 
However, as Siegel states mindfulness training is not the same as relaxation training. You may feel calm or you may not feel calm, but it's a form of mindful awareness which is very different from just relaxing (Dan Siegel, 2009, retrieved from: https://www.youtube.com/watch?v=Gr4Od7kqDT8)

The distinction I would like to make between colouring in and creating one's own mandala, is that colouring in may well be a form of relaxation but creating one's own work can be a mindful, expressive, unique expansion of oneself. I feel producing an original mandala is more complex and integrates both sides of the brain which can bring about well being and harmony within. I feel when I use predominately the right side of my brain utilizing descriptive attitude and my imagery, in conjunction with the left side of the brain where I use my verbalizing and logical thought process vocabulary. I bring together both sides of my brain in the quieting process of the conversation within a mandala. As Siegel (2009) states the key is to link left and right, not replace one imbalance with another (p.116). 

I do value the Buddhist, Native American Indian and Jungian viewpoints in relation to creating mandalas. However, in our western culture mandalas are not a formal teaching tool or used to foster cultural identity through the creative process. Sadly, mandalas have become a diagnostic tool as evidenced by the mandala assessment research instrument (MARIA). The process of creating a mandala has lost the quality to teach ourselves about ourselves as well as heal and restore aspects of ourselves. 
Reflecting on my own relationship to the process of creating a mandala I feel as though it is an embodied centering process. I agree with the Buchalter (2013) statement the mandala serves as a tangible object in which to focus one's inner thoughts, assisting in the difficulty of quieting one's mind during the practice of meditation (as cited in Paige. M. Beckwith, para 11). 
I intentionally focus on the present moment and attune myself to my embodied felt sensing. Emotions become more evident when I sit within my embodied self and when I quieten my reasoning judgmental mind. This is in contrast to striving to achieve an outcome, intentions are more about directing consciousness towards the desired change. They come from a deeper place (B. J. Davis, 2015, p. 44). The mandala process is not about problem solving, its about connecting, reflecting and making meaning. 

The mandala process is a mindful process, but also a reflective one. I believe our society does not place significant value on our emotional health, so hearing there is evidence that mindfulness helps develop effective emotion regulation in the brainÓ (Corcoran, Farb, Anderson, & Segal, 2010; Farb et al., 2010; Siegel, 2007) makes me value such mindful activities. Hence I feel treating mandalas mindfully is extremely important. As Jung (1968) states real liberation comes not from glossing over or repressing painful states of feeling, but only from experiencing them to the full (p.335). 

The creation of mandalas enables me to become calm and tranquil. Drawing the same line over and over becomes hypnotic and helps me focus on the present moment and not get caught in the explosion of my thoughts. In a single day we seem to have a thousand and one distractions making it hard to be still. But when I am free to focus within, and have a conversation with a mandala, I focus on my own experiences at a deeper level. As Lowith (1978) states As opposed to forgetting, repetition serves recollection of the meaning of existence, in order to opt into existence (p. 170). Hass-Cohen and Findlay mentioned: the round shape pulls the clients art making into the centre, promoting a sense of control and concentration. (check page, Noah Hass-Cohen, Joanna C Findlay, art therapy and the neuroscience of relationship, creativity and resiliency) cant find page number

I feel reflecting on the process is as much a part of the work with a client as generating the mandala. Siegel (2012)states: the notion of reflection is the awareness which brings the mind to a place, that goes beneath automatic pilot and which allows new combinations to be created, so that behaviours differ (Seigel, Retrieved: https://www.youtube.com/watch?v=CVYd1W4iAm0

My own reflective process allows me to see the surprising elements that come to the surface and serve as an access point for me to explore further. I am able to make sense of my embodied feelings and how I feel in that moment. I believe that art wakes us up from amoral sleep by helping us to feel more, see more, imagine more, contemplate more and know more (M. A. Franklin, 2012, p. 89). 

Writing down particular key words which come to mind from the mandala is also important. As Siegel (2009) states using words to describe and label this internal world can actually be useful .This is not only for people who have trouble accessing their emotions, but for those who need to find a way to bring balance to overactive feelings (p. 116). Hence titling and describing what I see, think and feel, and dialoguing with the image, helps me to integrate and attune into my inner world which promotes my resilience. Shifting my own perspectives by turning the mandala and viewing it from different angle and distances also helps to amplify what lays within the image. 

Consequently, I feel that emotional processing involves arousal, regulation, transformation, awareness and meaning making (Greeenberg et al. 2006, p293) And this mandala process allows us to make sense of our day, our thoughts and emotions. Drawing a mandala is a creative way to explore our emotions which may have been out of our awareness along with other pre reflective experiencing. Lett (1992) states using the arts in therapy helps us to uncover an unimagined storehouse of inner richness- packages from life put away into the spare room of disconnectedness, until an experiential imperative arises that drives us into our neglected selves. (how the arts make a different in therapy article)no page number para. 37). Taking an interpersonal Nero-Biological stance of awareness Siegel (2012) said when we use awareness to intentionally focus attention in a certain way, you change not just the function of the brain, but also you change its structure (Siegel, Retrieved: https://www.youtube.com/watch?v=CVYd1W4iAm0)



