\chapter{Amplification of clusters}


%FIGURE 1- DOUBLE EXPOSURE OF MANDALAS


The following clusters are only in relation to the previous mandala (1 to 6). 

I laid out all the mandalas on the floor and observe them altogether. I felt a huge connection but interestingly at the same time a disconnection. I have come so far personally with my emotional understanding. When I began drawing these mandalas I was heart-broken and extremely sad, confused, angry and bitter, but as I reflect now I feel at peace and confident within myself. The following are words I listed over a 5-month period to show the changes to my emotional life. 

In May 2015 I wrote the following words, I feel:

Rejected
Sad
Anger
Compassion
Sarcastic
Shame
Shock
Annoyance
Bitterness
Anxiety
Contempt
Resentment
Embarrassed
Concern
Sorrow
Astonishment
Dependence
Depressed
Defeated
Disappointment
Hurt
Hopeless
Fearful
Insulted
Pride
Jealous 
Loneliness
Abandoned 
Raw


In September 2015 I now feel:
Strong
Peaceful
Contemplative
Enthusiastic
Optimistic
Calm
Content
Alive
Inquisitive
Driven
Happy
Good
Honest
Curious
Joyful
Appreciative 
Wonder
I understand
Forgiveness 
Hopeful
Compassionate 
Warm
Powerful 
Confident 
Adaptable 
Soft
Resilient  


I clustered the words from the mandalas, created titles and then reduced each title to one or two words:
Mandala

Title cluster
Reduction title
\begin{enumerate}
\item Evolving into
Evolve
\item Fertilise the new direction 
Fertilise 
\item You have nothing to be embarrassed about 
Be proud
\item Moving the anger 
Move
\item Zone into connection 
Zone into 
\item Letting go of the why?
Let go
\end{enumerate}


I took a breath and now I see a larger theme within my conversations of each mandala. I see my personal journey of making sense of my relationship breakdown, as well as making sense of what my needs and values are. The mandalas I have drawn, even though it was some time ago, were each drawn as a unit without any expectation that they would tell a story. On reflection, I feel that the cluster titles have helped highlight the connections within these mandalas and my emotional progression. 
 
They have in fact been in order. 

\begin{enumerate}
\item I have evolved out of a relationship and back into my own space. 
\item I then fertilised the new direction I have and firmed up my values and made sense of them. 
\item I then accepted that I have nothing to be ashamed of and that instead I should be proud of my values, my needs and who I am.
\item I moved past the anger, grief and loss and the many unresolved questions and self-doubt. I have moved into the present and I am happy again. I am getting ready to move into something new. 
\item  I have made new connections and I now see the importance of connections in my personal and professional life. I want to surround myself with people who are inspiring and who are grounded. 
\item I have let go of my former partner and the hope he would return. I have let go of the intense anger, resentment and confusion and I have let go of all the unresolved questions. They are not important anymore. 
\end{enumerate}

It is so interesting to note that even though I was emotionally shattered, I happily sunk into the depths of my despair and the unknown of what the future holds for me. I appreciated that every ending has a new beginning. But it did not stop the physical and emotional pain I felt initially. After 2 months after my former partner left, I wrote the following at 4am:
	
You ripped my heart out and beat it until it went splat
You stripped my skin off until my bones were left
You plucked my eyes out and butchered my tears
You beheaded me without any emotion 
You assaulted my fears and raped my soul
And I am left wondering why? Why did you leave?

Daniel Siegel states the following and it explains that the loss of someone we love cannot be adequately expressed with words. Grappling with loss, struggling with disconnection and despair, fills us with a sense of anguish and actual pain. Indeed, the parts of our brain that process physical pain overlap with the neutral centres that record social ruptures and rejection. Loss rips us apart. (Siegel, p, 2010) 
(Mindsight: The new science of personal transformation, p find out from book? 2010) 

Thus, I really feel that my conversations within the mandalas gave me a platform to express and shift my emotional anguish. I felt held by remaining within the circles of the Mandalas. It was as though my conversations were connected to all parts of my being and these images that surfaced were keys to my new knowings. I feel as though all my mandalas have become my jewels and so they live within me and beyond. 






















