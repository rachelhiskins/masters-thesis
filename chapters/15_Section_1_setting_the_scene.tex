\chapter{Section 1 
Setting the scene}


Title: Self-Expression

When I started this inquiry I was so intent on addressing the therapeutic benefits of working with mandalas. I became lost in the therapeutic or deeply insightful moments that I constantly had when I was engaging in my own process. As I commenced writing these findings, I realized I had forgotten that I had taken myself as a therapist out of the picture as well as the relationship between client and therapist that proceeds the making of a mandala. It was as though I reduced the whole process to just the physical action of drawing a mandala. 

As I write this, my pattern of taking myself out of the equation has always been prominent in my personal and professional life. I realise that I tend to place others on a pedestal and never confidently trust my own abilities and skills. I down play my capacity, even though I am a very competent therapist and uniquely creative human being. I always thought that I was humble and I liked that, however I have come to believe this is a rather distorted way of thinking and it has clouded my confidence. I can now clearly see that being confident as a person and a professional is quite different to being arrogant. I feel now I can remain humble whilst being confident.

Before I address how I have used mandalas in the work I have done as a therapist, I firstly want to highlight that, with any therapeutic endeavour, building rapport and trust with the other is essential. I would offer the mandala process to the other only after this trust and rapport has been established. In my work I value multi- modal self-expression. The visual arts are where I feel most comfortable and connected. I find it easier to describe what I see, rather than explain what I think, and only then I can access my embodied feelings. Siegel (2009) states When we describe rather than explain, we are bringing the experientially rich right side into collaboration with the word-smithing left hemisphere (p. 114). Siegel also says that Tuning in to the body's signals and to imagery that arose from them also helped [his client]gain awareness of his feelings, because feelings themselves are the subjective sensation of what is going on inside the entire body- from our limbs and torso up to our brainstem, limbic areas and cortex (p. 115).

Through this inquiry I have come to know there are different ways of knowing, some of which are preverbal and pre-reflective. When introducing the idea of the mandala within the therapy session, I feel it is important to state that process of making the image is really important. This is, I believe, true for any creative arts therapy processes. A process of making a mandala is kinaesthetic and with a given focus on attending to the process, this allows the art maker to incorporate their embodied sensations and the mandala details as it takes form. As Hass-Cohen et al. (2008) state Expressing, experiencing and learning how to regulate affects can perhaps happen more easily through sensory integration activities and kinaesthetic movement associated with art therapy activities (p. 35).

I believe it is important to be focused before attempting to work with a mandala. The focusing facilitates self expression and so I begin with an invitation to participate in a short breathing exercise for both of us to quiet our minds, bodies and slow our breaths. Again as Seigel says: Controlled attention enables us to develop the self-regulation that allows new ways of responding to create new patterns of neural activity and contribute to neural plasticity (2006. P.153).

I have noticed that we tend to over think, and so I feel there is even more reason to place our reasoning and judgmental mind to the side and focus on the bodily sensations in the moment. As Hanson explains:
The brain has specialized circuits that register negative experiences immediately in emotional memory. On the other hand, positive experiences Ð unless they are very novel or intense  have standard issue memory systems, and these require that something be held in awareness for many seconds in a row to transfer from short-term memory buffers to long-term storage. Since we rarely do this, most positive experiences flow through the brain like water through a sieve, while negative ones are caught every time. (Rick Hanson, 2016, para.2).
If indeed our brains have a negative bias, I feel focusing on not having an agenda on what to draw may help free this bias we all have. When I draw I feel as if I am forming or creating things that move us changes and expands our perception. (Tufnell et al. p, 41,2004). Hence, focusing on the lines, shapes and colours allows this movement away from the clouded thinking we may have when we are stuck in a negative frame of mind. 

In my art and my therapeutic work, I am concerned with meaning making and now that feels in the present moment. I believe that there is no wrong way to create a mandala and there are no prescribed techniques that need to be followed, However, I feel drawing a comfortable size circle to begin to create a mandala is a good beginning. Working from inside out reduces the smudging and visually it is, I believe, satisfying to see how the mandala builds and expands.

When I companion another, I try to stay present and foster interpersonal attunement and endeavour to show my resonances all of which builds trust. I like to create a mandala alongside the other which also seems to foster a safe creative space and support. Once the other person has finished I generally place my mandala to the side and reflect on this on my own after the session. During the drawing phrase of a mandala with the other, I generally don't initiate a conversation unless the other person asks for support or makes comments that I feel require a response from me. 


