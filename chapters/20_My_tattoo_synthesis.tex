\chapter{My tattoo synthesis }


%FIGURE 6-ALL MY TATTOO DESIGNS LAYOUT


%My tattoo synthesis
My tattoo as a creative synthesis is a practical example of the power of the mandala process. As I reflected on my data, I became intent on designing a tattoo for myself. This fixation was driven by my exploration of the themes which came to light through my mandalas 1-6. A deep reflection on my process was what I needed to make sense of my new 'knowings'. 

Being present and grounded within myself has always been a strong value of mine and in 1996 I tattooed myself with a sewing needle and Indian ink (I simply pricked myself with some Indian ink on a needle.) The circle shape took me 1 hour to do. My design was supposed to reflect a circle which wrapped into itself on the top, representing balance within myself and the outer edge of the letter 'R' for Rachel directly underneath. 

%FIGURE 7-MY SELF INFLICTED TATTOO

It was six years later that I saw this tattoo in a new light, when my former partner and I started our relationship. We were both rather dumbfounded to see his name in my design when it was viewed from another angle. For many years I considered having a professional tattoo of his name as we saw it on my leg. It is only now that I have become utterly convinced that I should get another tattoo over this original one. I feel tattooing over the shape I created which reminds me of his name is symbolic of letting go and moving forward. 

Yes, a mandala tattoo! How fitting you might say. I am going to remain true to my original tattoo which was about balance, but will take this new design further to embody my 'knowings' about my needs and values. This mandala includes balance but also the value of 'reciprocity' which is so important to me at this point in my life. 

I love that through this process of quietening conversation with mandalas, I permanently engraved my skin with a mandala, which I feel indicates how much I respect the process. Having a mandala tattoo feels completely right and authentic.

On the 24th of September 2015 Kane Melbourne tattooed the mandala design on my ankle. I was excited and scared at the same time. It took around 45 minutes to complete. I was expecting it to be excruciatingly painful but I was pleasantly surprised. It did not hurt a great deal and the moments of pain I could easily withstand. The sound of the tattoo machine was not as loud and distracting as I thought and having background music helped to focus my mind away from those moments of pain. I was wanting to watch the process, but I was lying in an awkward position and so could not see. I did occasionally look to see Kane Melbourne chewing his gum in a rhythmic motion or singing along to the songs while he focused on his work. After payment, Kane Melbourne told me he really enjoyed tattooing mandalas and suggested to come back if I wanted another. After a few days I emailed Kane Melbourne as I was curious to know his experience of tattooing the mandala design on my leg. 

He responded quickly to say:  
As with most tattoos, especially for new clientele the process always begins with mild stress and anxiety. But once the stencil is where it needs to be and work gets underway the whole process becomes a bit Zen to be honest. You kind of get in the zone and everything else loses importance. Stippling dot work specifically is particularly calming.
I definitely found during the process that I was calmer. Between the process and the conversation, it was very chilled. From your arrival until about five minutes into the actual tattooing there is underlying situational analysing. Making sure you were comfortable and relaxed is secondary only to my state (since I'm the one who has to perform).

After the tattoo was finished I felt a sense of accomplishment I guess. You know, regardless of the design they're all your little babies. It's just that some of the designs speak more to me than others.

I do enjoy drawing and tattooing mandalas. Where other art can be constantly exerting creative pressure on you, once the design is laid out as a mandala there's a certain "happiness in slavery". You follow the lines, and then fill each section the same way. The repetition is calming. Depending on the design you may end up outlining and filling precisely the same shape fifty times. I also love that perfect symmetry that when you study it further it is made up of little pockets of chaotic dot work.

Answering all those questions definitely made me think about my process. It was an enlightening process.

I feel like this tattoo should have always been there. It does not feel alien having this new ink on my skin. A couple of days after my tattoo, my foot was achy, but the actually tattoo feels right at home. I am proud of my design; the meaning and the story behind it. I have noticed that so far I have kept the meaning to myself and I feel content with not sharing the significance with too many people. Actually I feel protective of it and I am really happy for other people just to observe its beauty and not know the long journey from the first tattoo through the inquiry process, to this lovely mandala. 
 
I designed the outer designs of the tattoo to be different and I specifically wanted them to reflect the diverse moments in life as we all change. I purposely made no connection to the meaning of these symbols, but I feel having Kane Melbourne reduce these symbols also supports the idea of needing a fixed understanding. I love that I left some unknowns in my mandala because like life there are so many unknowns. 

%FIGURE 8-BEFORE

%FIGURE 9-AFTER GETTING THE NEW TATTOO ON TOP OF THE OLD



