\chapter{Incorporating my participants into my inquiry}


%FIGURE 1- INTERSUBJECTIVE RESPONSE FROM A PARTICPANT AFTER THE MANDALA WORKSHOP

%Incorporating my participants into my inquiry
Running a workshop with the general public was not feasible given the time line for this project so instead, I ran a Workshop on the 14/6/15 with fellow students in the Masters students at the MIECAT Institute. Ethical guidelines were adhered to. This workshop was aimed at exploring the experience of drawing a mandala. We began with drawing and later explored further, using the MIECAT inquiry the meanings related to this exploration. 

\begin{itemize}
\item 
The group was composed of 4 participants, including myself.
\item Each participant gave permission for their artworks to be photographed and used in the thesis 
\item No images of the participants are to be used in the thesis
\item Voice recordings are confidential and only used for the researchers to check details 
\item Each participant is identified by a chosen pseudonym. 
\end{itemize}

The pseudonyms are as follows:

\begin{enumerate}
\item India
\item Leah
\item Inala
\item Dooly
\end{enumerate}

\subsection{The session:}

Our session was scheduled to start at 1:30 but by the time we all sat down, it was 1:40 pm. On the table were a few of my mandalas which I had drawn free hand with pen to give some examples of what I have been creating. From 1:45-1:50 I showed the participants the mandalas which I had previously drawn. I explained that many of them took me 2-3 hours over many sittings. I asked them to keep this in mind and note that we did not have that much time. I also asked my participants to keep in mind that they would each be creating a unique and amazing representation and encouraged them not to compare work. I encouraged the participants to use any art medium they felt comfortable with, as well as stating that there was no need to replicate my process of creating a mandala. However, in saying that, all participants chose to work using my own technique of drawing a mandala.

I asked India, Leah, Inala and Dooly to be present to their thoughts, feelings and bodily sensations whilst in the process of creating a mandala. I asked that they give importance to emergence and not to execute the planned mandala. I shared that my process is very emergent and that the supposed mistakes I make or something that just does not sit well with me often leads to something I prefer after I work further with the the marks. I gave an example where a mistake did not feel right for me in one of my mandalas, and showed how I added to this section, creating more vibrant marks which depicted that moment in time for me. 

I noticed myself feeling rushed at the beginning of the workshop and slightly embarrassed about speaking in front of the small group, as well as displaying my artwork. I planned the time but then I was confused and asked how long we had left for the session. As a result of my confusion, I may have sped up the process. However, each participant stated that it was the right amount of time, even though no one really finished their mandala. Most participants noticed themselves wanting to take the time and so they did not want to rush, hence an extra 15 minutes would not have made a difference. All participants were happy to end the session with mostly unfinished mandalas, but inspired to continue their mandalas at their leisure. 
2:45-3:25
We discussed the participants reactions to drawing a Mandala whilst keeping track of thoughts, feelings, emotions and bodily sensations. We finished and packed up at 3:25pm. 

\subsection{Participants experiences}

All the text below is taken from the session recording or from my notes taken in the debriefing of the mandala session and will be shown in the following font: Baskerville old face 

\subsection{India experience:}
%FIGURE 2-INDIA STARTING HER MANDALA
I was trying to replicate and do my own version of gorgeous and beautiful. So when I started it did not feel genuine, as I was wanting to make it beautiful and figurative
But then I felt inspired by parts of your mandalas that you had to change as they didn't feel right
By the time I started from the inside out to the edge I had found something that felt natural and felt right and then I felt I had to go back and attend to the centre again and fix it up
The first mark on the page I wrote- Endless opportunity, new journey, fresh face, back pack stacked with impractical travel products, before the heels become callus and the gut Para genetic
The process of drawing and tracking our Mandala was life giving
I was so excited to start the process but I didn't realise the depth. It was like a black hole; I was sucked into it.
I was thinking how mandalas are naturally occurring in life. Looking through the top, it has a ring around which looked very much like a mandala.
%FIGURE 3-INDIAS CRYSTAL CUP OF TEA MANDALA

My mandala reminds me of a Dandelion. If I set out to draw a Dandelion I would not have been able to and it would have been a fraud process. But it's not necessarily a dandelion

%FIGURE 4- INDIA COMPLETED MANDALA

%FIGURE 5-A DANDELION

\subsection{Leah experience:}
%FIGURE 6-LEAH MANDALA USING A WATER SOLUABLE PEN

I love this pen, and it is a water soluble pen, so I really wanted to include that quality, but then at the end I was like, shit we are out of time, let me put water in, but then I felt I may have ruined it by doing that
I was so focused on drawing the mandala, I lost focus on keeping track of everything
I feel the process was similar to mindful focusing.
I made circles and thought of a drawing of a few weeks back around my work places because both of the places I work their logos encompassed in a circle and I started doing those and I thought these again.
Feeling unfinished, Felt really rushed at the end. I could have kept going for a couple more hours.
Inala experience:

\subsection{Inala}

%FIGURE 7-INALA STARTING HER MANDALA

%FIGURE 8-INALA MANDALA AT THE END OF THE WORKSHOP
I wanted to focus on the emotion of being uncomfortable or unsettled and if things came out of that, that I could use in my own thesis, all the better.
I realised that as soon as I gave myself permission to not worry about things for my thesis, things were coming in anyway and it was about, not necessarily my experience in work, but more about keeping safety and innocence amongst danger and chaos; striving for safety and a sense of detachment from the outside experience to the present session. And that became a lot of the symbols, but then I had sounds stuck in my head and the symbols came from the sound. The circle had the weewaa sound.
It was really nice to attach meaning to it and also have the freedom to not be this profound. It was really nice to have the opportunity to tune into to that stuff, as I am aware that my brain is always going but never aware of where it's going, it's always busy but I have never thought to write down the stream of thoughts of consciousness.

\subsection{Discussion after Workshop with group:}

I posed the question how do I name my inquiry? Experience of making a mandala or maybe drawing into knowing through creating a mandala?

Inala: 
It is like you are having a dialogue within the Mandala, you are having a conversation with it, you were asking it questions and it was asking things back at you and you were writing things down the conversation when it was happening 

India:
Choices and decisions you make at every single moment. Deciding how to mark the page. Decision after decision and then the judgment of decisions, or giving self-permission

Inala:
And in that relationship, it is forgiving you and you are forgiving it and all the little mistakes and being like that does not look like the one next to it, but that's alright and then in the whole picture it forgives you back. 

Leah:
Ya, it does seem like a two-way relationship. 

Dooly:
Ya those imperfections really make it because it is a hand drawn mandala and you are not expected to get it perfect and thatÕs what I like about it. Ya it's not mechanical 

Leah:
Maybe it's not the symmetry that brings beauty, or maybe it's the asymmetry that give it personality. 

Inala:
Consistency does not have to mean symmetry. Trying to keep a consistent pattern but letting go of the fact they are not going to be the same

India:
My very first line is- already an imperfect circle with the compass.
I feel like this is similar to what our supervisor said write your way into knowing, but this is drawing your way into knowing.

We all agreed the process had a transformative quality. It was like a quietening conversation within a mandala, being able to rest in it, reflecting and emptying out our thoughts. All were touched by the ambient music I played in the background and the overall process of creating their own mandalas.














