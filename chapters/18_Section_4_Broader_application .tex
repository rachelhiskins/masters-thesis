\chapter{Section 4
Broader application
What is good for clients, is good enough for therapists}


%FIGURE 1-SELF-CARE- TITLED: STOP, LOOK WITHIN, STRETCH IN THE POSSIBILITIES


Huang-po, ninth-centery Chinese Zen master said "

"Chase it and it always eludes you; run from it and it is always there" (McNiff, 2004, p. 121). 

Siegel states that reflections and being aware of our intersubjective lives is essential for our well being. He calls this time in where we refer inward to our inner worlds and sift through our sensations, imagery, feelings and thoughts. Through this thesis inquiry I reflected and remain curious on the questions of what are emotions and how are they formed. Siegel (2011) said 
"the take home message are emotions arise from the body, they interact in the lower part of our brain and it's this higher part, behind your forehead, that coordinates and balances the whole thing" (Dan Siegel, Retrieved: https://www.youtube.com/watch?v=J-BJpvdBBp4)

Siegel (2012) also states " when we create integration in our lives through using mindfulness and creative activities we can better reflect and have relationships which connect and are caring this stimulates the growth of the integrative fibres of the brain, and these fibres allow you to have resilience." (Siegel, retrieved: youtube.com/watch?v=LiyaSr5aeho)and the internal education where you are getting this self connection directly helps your connection with others (Siegel, 2009, retrieved: https://www.youtube.com/watch?v=Gr4Od7kqDT8) As a therapist it is fundamental to our clients and our own well being that we connect, reflect and take care of ourselves. 



