
\chapter{Section 3 
The Emergence and re-looking process of a mandala}

%Emergence- Titled: Feeling within the wobbles

The mandala process is an emergent process. I feel OK with the doubts and wobbles and trust that the journey of co-creation within mandalas, assists us to experience all of our experiencing in a more fulfilled and meaningful way. This approach is similar to that described by McNiff when (1998) he states trusting the process is based on a belief that something valuable will emerge when we step into the unknown (p. 27) 

I really feel as though I trusted my process of stepping into the conversations within mandalas, knowing that something important come to light from my experience. As McNiff (1998) says withhold your desire for outcomes. You don't want to be at the end of the process when you are just beginning or in the thick of it (p 116). Of course there was a part of me that just wanted to skip to the end where I understood everything and felt good again. But rather than blunting the awareness of my conflicts through avoidance, I try to stay close to them, to directly engage their power to transform. (McNiff, 2004, p. 54). 

To reiterate, this mandala process is not a problem solving activity. Even though, in the end it may have this quality, it is rather a process that allows the mind [to be]is a self organizing emergent process, that is both embodied and relational, which regulates the flow of energy and informationÓ (Dr Dan, 2012, Retrieved: https://www.youtube.com/watch?v=CVYd1W4iAm0 

