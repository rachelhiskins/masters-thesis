\chapter{My findings}

It is important to say that writing my thesis has been a therapeutic endeavour. As a result, I have grown and feel I have become more authentic in the way I relate to others and to myself. Importantly, as I inquired into the experiences of drawing mandalas, I actually experienced significant benefits. Like Jung, I too experienced a transformation of sorts through this process. Jung also used mandala drawing as a way of making sense of his experience. He wrote:

\begin{quotation}
I sketched every morning in a notebook a small circular drawing which seemed to correspond to my inner situation at the time. With the help of these drawings I could observe my psychic transformation from day to day
      Only gradually did I discover what the mandala really is, Formation, Transformations...
      Mandalas were cryptograms concerning the state of the self which were presented to me anew everyday. In them I saw the self- that is my whole being-actively at work. To be sure, at first I could only dimly understand them, but they seemed to me highly significant, and I guarded them like precious pearls (Jung, 1963, P. 187).
\end{quotation}

In order to form my findings in a coherent manner, I returned to the inquiry process define my emerging understanding of the power of mandala drawing in therapeutic endeavours. To this end I selected keywords that most strongly resonated with me in my own creative process of drawing mandalas that I have previously documented, as well as the resonant key phrases from the participantsÕ reflections. I clustered these phrases and created titles for each cluster. I also drew a mandala for the 10 clusters titles. I became aware that each mandala is illustrative of the therapeutic work I do. The following are the phrases and key words that were most resonant:
(The different fonts delineate my reflections in ...and the participants reflections shown in ...font.)


\begin{enumerate}
\item Repetition
Repetition is calming
\item Mindfulness
Trying to keep a consistent pattern when drawing but letting go of the fact that each line will never be the same
Practice listening to our emotions, thoughts, feelings, as well as our bodies
Intuitive
\item Processing
Use no measurements, just my felt sense when I draw
Having a break helps to process and evaluate where my emotions and thoughts are
\item Hypnotic/ Focusing
Fixed on the process and not the end result- concentrating and focusing on each line
I became lost in the stillness at times
\item Emergence
Symbology seems to emerge within the process without thought
\item Decelerate/ Calming
Slowing down all my thoughts and feelings
\item Self expression
Providing me with a dialogue to better understand my unconsciousness and connect me with my own visual language
\item Connections/Awareness
Fascinated by what it might have to teach me
What the symbols represents as well as what resonates for me now. As well as becoming aware of our bodies, mind and thoughts
\item Self care
Debriefing tool
Self care
Difference perspective
Connecting to our emotions
\item Relief
A sense of relief from my emotional content
Self-reflective
The process has a transformative quality- which I feel heals old patterns and emotional wounds
I felt held by remaining within the circles
\end{enumerate}

To create a more focused findings sections of this thesis, I decided to combine some clusters, thereby reducing the number of clusters from 10 to 5. Each of the 5 sections that follow include the cluster titles, the mandalas and references to the relevant literature.  

\underline{Section 1}

Mandala: 
Self expression (bold) 

Section 1 sets the scene for how a mandala can be introduced within the therapeutic setting. I feel self expression is integral to arts therapy practice, but also to each of us, whether this expression occurs visually or in other modalities. I did not want to cluster this mandala with any other. 

\underline{Section 2}

Mandala:
\begin{itemize}
\item Repetition
\item Mindfulness
\item Processing 
\item Hypnotic/ focusing 
\item Decelerate/calming
\item Connection/awareness 
\end{itemize}


Section 2 speaks about how the making of a mandala is an embodied action. I clustered the 6 mandalas above as there is a similarity in the way each speak of embodied action with the mandala.

\underline{Section 3}

Mandala:
Emergence

In section 3 I address the emergence and importance of re looking within the process of working with mandalas in a therapeutic context.

\underline{Section 4}

Mandala:
Self care 

In section 4 I describe the broader application of self care and the value for the therapist in practicing ways of regulating their own emotional experience. As therapists, we encounter exceedingly energetic emotions in arts therapy sessions, we need to take care of ourselves, so we can better serve the other.

\underline{Section 5}

Mandala:
Relief

In section 5, I conclude my thesis by coming back to my experience of my embodied and mindful conversations within mandalas. I give a practical example of my journey of experiencing a relief from distress and un ease. I also describe my process of getting a tattoo. I feel this process was a creative synthesis; an expression of my embodied knowing. 

