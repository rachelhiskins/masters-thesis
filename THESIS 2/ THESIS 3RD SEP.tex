\documentclass[oneside]{book}
\usepackage{graphicx}
\usepackage[utf8]{inputenc}
\usepackage[paperheight=210mm, paperwidth=210mm, margin=2cm, heightrounded]{geometry}
\usepackage{calligra}
\usepackage[T1]{fontenc}
\usepackage{setspace}
\author{Rachel Hiskins}
\calligra
\title{{\Huge A quiet conversation within a mandala}\\[15pt]
\space
\texttt{An inquiry into the therapeutic \\benefits of creating a mindful mandala}}

\date {December 2016}

%\begin{description}\item[ Submitted in partial fulfilment of the requirements\\for the degree of Masters of Therapeutic Arts by Supervision \\The MIECAT Institute, Melbourne, Australia Statement of Authorship] \end{description}

\makeindex
\begin{document}
\begin{titlepage}
	\calligra
    \maketitle
\end{titlepage}
\doublespacing 

\tableofcontents  


%Introduction (chapter)
	\input{../chapters/2-Introduction.tex}

\chapter{Background}
\begin{quote}
"Jung states a Man cannot stand a meaningless life" (C Jung) 
\end{quote}

\newpage
\begin{figure}[h!]
\begin{center}
\includegraphics[scale=.25]{../eps/In_through_the_background.eps}
\caption{In through the background}
\label{label}
\end{center}
\end{figure}

\newpage
Before I start my inquiry, I want to orient you the reader and provide a small window illustrating the complexity of what mandalas can pose and why I hold drawing mandalas so dear to my heart. 

I sat down to reflect on the mandalas I have kept over the years and one stood out for me. I sat with this mandala which I produced in 2005 and was astounded by the message it had for me 10 years after producing it and I still remain fascinated by what it might continue to teach me in the future. I continue to be curious in particular about the power of mandalas and what the symbols represents as well as what resonates for me now. 

I explored two mandalas which were created 10 years apart, but remarkably have a connection and seem to be unified.   

The following mandala I drew on the 22 of November 2005 (which was the third year out of the twelve years my former partner and I were together) I drew this mandala in response to a dream amplification. 

\begin{figure}[!thbp]
\begin{center}
\includegraphics[scale=.1]{../eps/2005_m.eps} 
\caption{2005 Mandala}
\label{figure}
\end{center}
\end{figure}

%FIGURE 2-2005 MANDALA (try and get this up in this placement 

I wrote the following in my journal: \newline 

I have many dreams where we have broken up, as he has cheated on me, however I never thought he would actually cheat in real life. I feel that this dream is an indication of my fear of abandonment. We have been together for 3 years. He has never wanted to comment on our future which bothers me as it poses the question- When will he break up with me? Perhaps my ego is trying to balance my relationship and my emotions. I feel a lack of attention from him, not being able to ease my emotions about being together and not knowing when he might pull the rug from under me. I also feel my dream is also trying to communicate that I need to break away from a situation or any bad habits that I may have. As well as being balanced within myself and accept myself and stop with the self-deprivation.  

It is amazing that 10 years after I produced this mandala in 2005, my former partner did pull the rug from under me. He ended our relationship in 2014 abruptly without warning. He withdrew and disappeared to never say another word to me. I still remember how I felt in the dreams I had when he broke up with me. All those years later, when he did end our relationship, he behaved exactly like in my dreams. He was cold, empty, no compassion and had totally withdrawn. The only thing he revealed was that he felt lost and did not know what he wanted. I, on the other hand, had very clear goals; I wanted to complete my masters and to start a family with him. But after 12 years he was just as confused when I met him regarding what his aspirations were. I was attracted to his intelligence, his many talents and his beautiful blue eyes. He was my best friend and lover for 12 years and never spoke about being unhappy in our relationship, but at the same time he never commented on how he saw our future, even when we decided to get engaged in late 2013. 

Reflecting on the phrase of pulling the rug from under me, I have only now been able to understand the significance of this symbolic rug. It now makes perfect sense why I felt it was so important for me to keep the beautiful rug we bought in India. We did not have many possessions but the rug was the only thing that I yearned for when we went our separate ways. I felt lost when he left as it was excruciating and so sad that he chose to walk away and vanish. However, in my grief, I remained grounded and I never lost sight of who I was. Thus, it was invaluable for me to put my own feet on this beautiful soft rug and not get caught up in my self doubt around the questions why he left.


\begin{figure}[htbp]
\begin{center}
\includegraphics[scale=.2]{../eps/My_rug_.eps}
\caption{MY RUG!}
\label{label}
\end{center}
\end{figure}
%my rug! 


Fast forward to the 8th of August 2015 to intensive three class for the MIECAT Masters program; after exploring my embodiment, I intuitively created a mandala which served as my new mantra. 

\begin{figure}[htbp]
\begin{center}
\includegraphics[scale=.05]{../eps/mandala_mantra.eps}
\caption{Mandala mantra}
\label{label}
\end{center}
\end{figure}
%FIGURE 4-MANDALA MANTRA (fix image, was a RAF file

\begin{figure}[htbp]
\begin{center}
\includegraphics[scale=.05]{../eps/Mandala_definition.eps}
\caption{Definition of the mandala mantra}
\label{label}
\end{center}
\end{figure}

%FIGURE 5- DEFINITION OF THE MANDALA MANTRA
%Get better picture and size

After my former partner revealed he wanted to separate, I woke up with the word reciprocity stuck in my head which has evidently stayed implanted in my body. As soon as I drew this wiggled line, I knew it represented reciprocity and balance for me. This wiggled line is similar to the shape of an 'S'. The wiggled shape is going through five different coloured circles, which are the same on either side of the wiggled line. They overlap each consecutive circle which is encompassed within a large circle. Each of these circles represent qualities I desire in an intimate relationship. 

They are as follows:
\begin{itemize}
\item Affection (white)
\end{itemize}

\begin{itemize}
\item Affirmation (red)
\end{itemize}

\begin{itemize}
\item Connection (green)
\end{itemize}

\begin{itemize}
\item Acknowledgment (yellow)
\end{itemize}

\begin{itemize}
\item and for the both of us to be Grounded (brown)
\end{itemize}

    
To my surprise this wiggled line is also similar to the centre of the mandala I drew all those years earlier in 2005, although in this early mandala this wiggled line is in the opposite direction. In the 2015 mandala the wiggled line has two different colours- red and green that slightly overlap each other in the centre of the line. In 2005 there is only black and white. This is quite interesting as in the past I looked at the world with a very black and white mentality but now I can see the world has so many more colours that blend into each other. It is refreshing to be able to appreciate their ever shifting elements and not be stuck in this fixed terminology. 

\begin{figure}[h!] 
\begin{center}
\includegraphics[scale=.1]{../eps/2005_m.eps}
\caption{2005 Mandala}
\label{label}
\end{center}
\end{figure}


%FIGURE 6- 2015 MANDALA MANTRA
\begin{figure}[htbp]
\begin{center}
\includegraphics[scale=.05]{../eps/mandala_mantra.eps}
\caption{Mandala mantra}
\label{label}
\end{center}
\end{figure}

%FIGURE 7- MANDALA 2005

Take a double exposure picture of these two mandalas



	%Background into my inquiry(section)

	
	%Methods and Methodology (chapter)
	
	
	\input{../chapters/4_Methods_and_Methodology.tex}

	%Methodology (section)
	%Methods (section)
	%Procedures (section)
	%Maps (section) 
	%Paradigm alignment (section}
	
%Values (chapter)

\chapter{Values}

\begin{quote}
"Who looks outside, dreams; who looks inside awakens." Jung 
\end{quote}
\newpage

\begin{figure}[!h]
\begin{center}
\includegraphics[scale=.3]{../eps/My_values.eps} 
\caption{My values}
\label{label}
\end{center}
\end{figure}

%FIGURE 1-MY VALUES 

Values

I have come to a better understanding of my values and needs and have become wiser for it, through creating my conversations within the mandalas. It has been great to be able to witness my own emotional evolution. I feel that I have healed parts of myself that needed attention. I feel I have immersed myself within the mandalas in combination with other processes. My values and needs were submerged before starting my inquiry. 

My titles are formed when I look at the mandala and rely on my felt sense, as well as any imaginative visualisations that might spring to mind. It is very much a free writing exercise with my visual imagery which comes to the fore.

\begin{figure}[!h]
\begin{center}
\includegraphics[scale=.12]{../eps/Double_exposure_of_my_values.eps} 
\caption{Double exposure of my values }
\label{label}
\end{center}
\end{figure}

%FIGURE 2- DOUBLE EXPOSURE OF MY VALUES
\newpage
\section{Connection}

\begin{figure}[!h]
\begin{center}
\includegraphics[scale=.2]{../eps/Connection-valueA.eps} 
\caption{Overlap, touch and sway in the wind}
\label{connection}
\end{center}
\end{figure}

%FIGURE 3-CONNECTION

The first value I have in my work and life is connection. It is nice to be awakened and realise that we live in a world where we are relational beings and connecting is an innate trait. As Buddha states 
\begin{quote} "All things appear and disappear because of the concurrence of causes and conditions. Nothing ever exists entirely alone; everything is in relation to everything else." 
\end{quote}

\newpage
\section{Clarity} 

\begin{figure}[!h]
\begin{center}
\includegraphics[scale=.2]{../eps/clarity.eps} 
\caption{Being able to understand peacefully}
\label{Clarity}
\end{center}
\end{figure}

%FIGURE 4-CLARITY

\begin{quote}
"If you want to understand the jungle, you cannot be content just to sail back and forth near the shore. Your'e got to get into it, no matter how strange or frightening it may seem."
\end{quote}

I value clarity, which is the quality of being able to easily see, hear and understand and to be able to understand not only the meaning within my experiences but the journey I took to arrive at these meanings. My thesis takes you through my process of finding clarity for myself and also describes why I value drawing mandalas from a therapeutic point of view. 

\newpage
\section {Imaginative visualisations}

\begin{figure}[!h]
\begin{center}
\includegraphics[scale=.2]{../eps/imaginative_visualizations_.eps} 
\caption{ So much to learn, open and see}
\label{Imaginative visualisations}
\end{center}
\end{figure}

%FIGURE 5-IMAGINATIVE VISUALISATIONS

My third value is about the power of imaginative visualisations. Elkins states that \begin{quote}
"The conscious mind engages in a more logical thought and rational thinking. The unconscious mind processes information in a more experiential manner and responds to feelings, images and suggestions more automatically" (pp59 Gary Elkins, )date.)
\end{quote}



Hypnotic relations therapy principles and applications) Hence, if we are aware that there is so much beneath the surface which is not conscious, and allow ourselves to be open to our imaginations, we may bring into awareness the little gems that have been outside our consciousness through this experiential explorations. Thus, I agree with McNiffs comment when he states: 

\begin{quote}
"I prefer a more imaginative attitude that views the image as a step a head of the reflecting mind, as a guide who shows me where I can go and what I can be." (McNiff, p 67,2004)(Art Heals) 
\end{quote}

\newpage
\section{Emergence}


\begin{figure}[!h]
\begin{center}
\includegraphics[scale=.1]{../eps/Emergence_A.eps} 
\caption{Openness to let things naturally form }
\label{Emergence}
\end{center}
\end{figure}

%FIGURE 6-EMERGENCE

My fourth value is emergence. To be able to let go and trust the process of emergence which in short is an intuitive improvisational quality and seems to derived from a felt resonance to the current moment.(Lett,p, 11, 2011)(making sense of our lives) As Merleau-Ponty states: 
\begin{quote}
"Thinking a movement is destroying the movement." (Merleau-Ponty 1962)
\end{quote}
 and so when we let our reflexive-reflective interactions flow into awareness we reap the benefits.








	%Connection
	%Clarity
	%Imaginative visualisation 
	%Emergence

%Mandala 1


\input{../chapters/6_Mandala_1.tex}

%Mandala 2
%Mandala 3
%Mandala 4
%Mandala 5
%Mandala 6

%Amplification of clusters

%Incorporating my participants into my inquiry

%My findings (chapter) 
	%section 1
		%setting the scene
	%section 2
		%The making of an embodied mandala 
	%Section 3
		%The emergence and re-looking process of a mandala
	%section 4
		%Broader application
	%section 5
		%practical example of self care
		%My tattoo sybthesis
		%tattoo process


\end{document}


